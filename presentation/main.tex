%! Author = BRAHIMA FOFANA
%! Date = 10/01/2026

% Preamble
\documentclass[11pt]{beamer}

% Thème
\usetheme{Madrid}
\usecolortheme{default}

% Packages
\usepackage[utf8]{inputenc}
\usepackage[french]{babel}
\usepackage{amsmath}
\usepackage{amssymb}
\usepackage{graphicx}
\usepackage{listings}
\usepackage{xcolor}
\usepackage{algorithm}
\usepackage{algorithmic}

% Configuration des listings Python
\lstset{
    language=Python,
    basicstyle=\tiny\ttfamily,
    keywordstyle=\color{blue},
    commentstyle=\color{gray},
    stringstyle=\color{red},
    showstringspaces=false,
    breaklines=true,
    frame=single,
    numbers=left,
    numberstyle=\tiny\color{gray}
}

% Informations du titre
\title[Méthodes Numériques]{Méthodes Numériques pour la Résolution d'Équations Différentielles et l'Intégration Numérique}
\author{BRAHIMA FOFANA}
\institute{Master 2 Génie Informatique}
\date{12 janvier 2026}

% Document
\begin{document}

% Page de titre
\begin{frame}
    \titlepage
\end{frame}

% Table des matières
\begin{frame}{Plan de la présentation}
    \tableofcontents
\end{frame}

% ========================================
% SECTION 1 : INTRODUCTION
% ========================================
\section{Introduction}

\begin{frame}{Contexte et Motivation}
    \begin{block}{Problématique}
        Les problèmes scientifiques et d'ingénierie nécessitent souvent :
        \begin{itemize}
            \item La résolution d'équations différentielles ordinaires (EDO)
            \item Le calcul d'intégrales définies complexes
        \end{itemize}
    \end{block}

    \begin{block}{Limites des solutions analytiques}
        \begin{itemize}
            \item Nombreuses EDO sans solution fermée
            \item Intégrales impossibles à calculer explicitement
            \item Nécessité d'approches numériques
        \end{itemize}
    \end{block}
\end{frame}

\begin{frame}{Objectifs du projet}
    \begin{enumerate}
        \item \textbf{Équations différentielles :} Implémenter et comparer trois méthodes à pas séparés
        \begin{itemize}
            \item Méthode d'Euler (ordre 1)
            \item Méthode de Heun (ordre 2)
            \item Méthode de Runge-Kutta d'ordre 4
        \end{itemize}

        \vspace{0.3cm}

        \item \textbf{Intégration numérique :} Évaluer cinq méthodes de quadrature
        \begin{itemize}
            \item Quadratures de Gauss (Laguerre, Legendre, Chebyshev)
            \item Règle de Simpson composite
            \item Intégration par spline quadratique
        \end{itemize}
    \end{enumerate}
\end{frame}

\begin{frame}{Problème test pour les EDO}
    \begin{block}{Équation différentielle étudiée}
        $$
        \begin{cases}
        y'(x) = \pi \cos(\pi x) \cdot y(x), & x \in [0, 6] \\
        y(0) = 1
        \end{cases}
        $$
    \end{block}

    \begin{block}{Solution analytique connue}
        $$
        y(x) = e^{\sin(\pi x)}
        $$
    \end{block}

    \vspace{0.2cm}

    \textbf{Avantage :} Permet de calculer l'erreur numérique précisément en comparant avec la solution exacte.
\end{frame}

% ========================================
% SECTION 2 : MÉTHODES EDO
% ========================================
\section{Résolution d'équations différentielles}

\begin{frame}{Méthode d'Euler}
    \begin{block}{Principe}
        Approximation linéaire de la solution à partir de la dérivée :
        $$
        y_{n+1} = y_n + h \cdot f(x_n, y_n)
        $$
        où $h$ est le pas de discrétisation.
    \end{block}

    \begin{block}{Caractéristiques}
        \begin{itemize}
            \item \textbf{Ordre de convergence :} $\mathcal{O}(h)$
            \item \textbf{Simplicité :} Méthode la plus simple
            \item \textbf{Précision :} Faible, nécessite un pas petit
            \item \textbf{Coût :} Une seule évaluation de $f$ par itération
        \end{itemize}
    \end{block}
\end{frame}


\begin{frame}{Résultats - Euler avec $h=0.3$}
    \begin{figure}
        \centering
        \includegraphics[width=0.75\textwidth]{figure/equa_diff/euler/euler_1.png}
        \caption{Méthode d'Euler avec $h=0.3$}
    \end{figure}

    \textbf{Observation :} Écart visible entre solution numérique et exacte, particulièrement pour les grandes valeurs de $x$.
\end{frame}

\begin{frame}{Résultats - Euler avec $h=0.15$}
    \begin{figure}
        \centering
        \includegraphics[width=0.75\textwidth]{figure/equa_diff/euler/euler_2.png}
        \caption{Méthode d'Euler avec $h=0.15$}
    \end{figure}

    \textbf{Observation :} Réduction du pas améliore la précision, mais l'erreur reste notable.
\end{frame}

\begin{frame}{Résultats - Euler avec $h=0.06$}
    \begin{figure}
        \centering
        \includegraphics[width=0.75\textwidth]{figure/equa_diff/euler/euler_3.png}
        \caption{Méthode d'Euler avec $h=0.06$}
    \end{figure}

    \textbf{Observation :} Avec un pas très fin, la solution numérique se rapproche de la solution exacte, au prix d'un coût calculatoire accru.
\end{frame}

\begin{frame}{Méthode de Heun}
    \begin{block}{Principe}
        Méthode prédicteur-correcteur utilisant une évaluation intermédiaire :
        $$
        \begin{aligned}
        k_1 &= f(x_n, y_n) \\
        k_2 &= f\left(x_n + \frac{h}{2}, y_n + \frac{h}{2} k_1 \right) \\
        y_{n+1} &= y_n + h \cdot k_2
        \end{aligned}
        $$
    \end{block}

    \begin{block}{Caractéristiques}
        \begin{itemize}
            \item \textbf{Ordre de convergence :} $\mathcal{O}(h^2)$
            \item \textbf{Complexité :} Deux évaluations de $f$ par itération
            \item \textbf{Précision :} Meilleure qu'Euler pour un pas équivalent
        \end{itemize}
    \end{block}
\end{frame}

\begin{frame}{Résultats - Heun avec $h=0.3$}
    \begin{figure}
        \centering
        \includegraphics[width=0.75\textwidth]{figure/equa_diff/heun/heun_1.png}
        \caption{Méthode de Heun avec $h=0.3$}
    \end{figure}

    \textbf{Observation :} Meilleure approximation qu'Euler avec le même pas, grâce à l'ordre 2.
\end{frame}

\begin{frame}{Résultats - Heun avec $h=0.15$}
    \begin{figure}
        \centering
        \includegraphics[width=0.75\textwidth]{figure/equa_diff/heun/heun_2.png}
        \caption{Méthode de Heun avec $h=0.15$}
    \end{figure}

    \textbf{Observation :} Solution numérique presque indiscernable de la solution exacte.
\end{frame}

\begin{frame}{Résultats - Heun avec $h=0.06$}
    \begin{figure}
        \centering
        \includegraphics[width=0.75\textwidth]{figure/equa_diff/heun/heun_3.png}
        \caption{Méthode de Heun avec $h=0.06$}
    \end{figure}

    \textbf{Observation :} Excellente précision même avec un pas relativement grand.
\end{frame}

\begin{frame}{Méthode de Runge-Kutta d'ordre 4 (RK4)}
    \begin{block}{Formules}
        $$
        \begin{aligned}
        k_1 &= h \cdot f(x_n, y_n) \\
        k_2 &= h \cdot f\left(x_n + \frac{h}{2}, y_n + \frac{k_1}{2}\right) \\
        k_3 &= h \cdot f\left(x_n + \frac{h}{2}, y_n + \frac{k_2}{2}\right) \\
        k_4 &= h \cdot f(x_n + h, y_n + k_3) \\
        y_{n+1} &= y_n + \frac{k_1 + 2k_2 + 2k_3 + k_4}{6}
        \end{aligned}
        $$
    \end{block}

    \begin{block}{Caractéristiques}
        \begin{itemize}
            \item \textbf{Ordre de convergence :} $\mathcal{O}(h^4)$
            \item \textbf{Complexité :} Quatre évaluations de $f$ par itération
            \item \textbf{Précision :} Excellente, standard en calcul scientifique
        \end{itemize}
    \end{block}
\end{frame}


\begin{frame}{Résultats - Runge-Kutta avec $h=0.5$}
    \begin{figure}
        \centering
        \includegraphics[width=0.75\textwidth]{figure/equa_diff/range_kunta/range_kunta_1.png}
        \caption{Méthode de Runge-Kutta avec $h=0.5$}
    \end{figure}

    \textbf{Observation :} Même avec un pas relativement grand, RK4 produit une excellente approximation.
\end{frame}

\begin{frame}{Résultats - Runge-Kutta avec $h=0.3$}
    \begin{figure}
        \centering
        \includegraphics[width=0.75\textwidth]{figure/equa_diff/range_kunta/range_kunta_2.png}
        \caption{Méthode de Runge-Kutta avec $h=0.3$}
    \end{figure}

    \textbf{Observation :} Précision remarquable, courbes numériques et exactes presque superposées.
\end{frame}

\begin{frame}{Résultats - Comparaison supplémentaire}
    \begin{figure}
        \centering
        \includegraphics[width=0.48\textwidth]{figure/equa_diff/range_kunta/range_kunta_3.png}
        \hfill
        \includegraphics[width=0.48\textwidth]{figure/equa_diff/range_kunta/range_kunta_4.png}
        \caption{Analyses complémentaires de la méthode RK4}
    \end{figure}
\end{frame}

\begin{frame}{Comparaison des méthodes EDO}
    \begin{table}
        \centering
        \small
        \begin{tabular}{|l|c|c|c|}
        \hline
        \textbf{Méthode} & \textbf{Ordre} & \textbf{Éval. par pas} & \textbf{Précision} \\
        \hline
        Euler & $\mathcal{O}(h)$ & 1 & Faible \\
        Heun & $\mathcal{O}(h^2)$ & 2 & Moyenne \\
        Runge-Kutta 4 & $\mathcal{O}(h^4)$ & 4 & Excellente \\
        \hline
        \end{tabular}
        \caption{Comparaison théorique des méthodes}
    \end{table}

    \vspace{0.3cm}

    \begin{block}{Conclusions pratiques}
        \begin{itemize}
            \item \textbf{Euler :} Simple mais nécessite un pas très petit
            \item \textbf{Heun :} Bon compromis précision/coût
            \item \textbf{RK4 :} Méthode de référence pour la précision
        \end{itemize}
    \end{block}
\end{frame}

% ========================================
% SECTION 3 : INTÉGRATION NUMÉRIQUE
% ========================================
\section{Méthodes d'intégration numérique}

\begin{frame}{Quadrature de Gauss-Laguerre}
    \begin{block}{Domaine d'application}
        Intégrales semi-infinies avec poids exponentiel :
        $$
        \int_0^{+\infty} e^{-x} f(x) \, dx \approx \sum_{i=1}^n w_i f(x_i)
        $$
    \end{block}

    \begin{block}{Propriétés}
        \begin{itemize}
            \item $x_i$ : racines du polynôme de Laguerre $L_n(x)$
            \item $w_i$ : poids associés
            \item \textbf{Avantage :} Convergence exponentielle pour fonctions régulières
            \item \textbf{Limitation :} Applicable uniquement pour intégrales sur $[0, +\infty)$
        \end{itemize}
    \end{block}

    \textbf{Exemple test :} $\int_0^{\infty} e^{-x} x^2 \, dx = 2$
\end{frame}

\begin{frame}{Résultats - Gauss-Laguerre {\small $\int_0^{+\infty} e^{-x} x^2 dx = 2$}}
    \begin{figure}
        \centering
        \includegraphics[width=0.48\textwidth]{figure/integral/gauss_laguerre/gauss_laguerre_erreur.png}
        \hfill
        \includegraphics[width=0.48\textwidth]{figure/integral/gauss_laguerre/gauss_laguerre_temps.png}
    \end{figure}

    \textbf{Interprétation :}
    \begin{itemize}
        \item Le poids $e^{-x}$ est intégré dans la quadrature $\Rightarrow$ convergence exponentielle
        \item Fonction polynomiale $x^2$ : exactitude atteinte pour $n \geq 2$
        \item Temps de calcul négligeable et quasi-constant avec $n$
    \end{itemize}
\end{frame}

\begin{frame}{Quadrature de Gauss-Legendre}
    \begin{block}{Formule}
        Intégrale sur intervalle fini $[a, b]$ :
        $$
        \int_a^b f(x) \, dx \approx \frac{b-a}{2} \sum_{i=1}^n w_i f\left( \frac{b-a}{2} x_i + \frac{a+b}{2} \right)
        $$
    \end{block}

    \begin{block}{Caractéristiques}
        \begin{itemize}
            \item $x_i$ : racines du polynôme de Legendre $P_n(x)$
            \item \textbf{Propriété :} Exacte pour polynômes de degré $\leq 2n-1$
            \item \textbf{Avantage :} Polyvalente, applicable à tout intervalle fini
            \item \textbf{Usage :} Méthode standard pour intégrales régulières
        \end{itemize}
    \end{block}

    \textbf{Exemple test :} $\int_{-1}^1 \cos(x) \, dx = 2\sin(1)$
\end{frame}

\begin{frame}{Résultats - Gauss-Legendre {\small $\int_{-1}^{1} \cos(x) dx = 2\sin(1)$}}
    \begin{figure}
        \centering
        \includegraphics[width=0.48\textwidth]{figure/integral/gauss_legendre/gauss_legendre_erreur.png}
        \hfill
        \includegraphics[width=0.48\textwidth]{figure/integral/gauss_legendre/gauss_legendre_temps.png}
    \end{figure}

    \textbf{Interprétation :}
    \begin{itemize}
        \item Fonction analytique sur intervalle fini : convergence exponentielle observée
        \item Précision machine atteinte pour $n \approx 10$ (erreur $< 10^{-15}$)
        \item Temps de calcul très compétitif malgré le calcul des racines de $P_n(x)$
    \end{itemize}
\end{frame}

\begin{frame}{Quadrature de Gauss-Chebyshev}
    \begin{block}{Formule}
        Intégrale avec poids singulier :
        $$
        \int_{-1}^{1} \frac{f(x)}{\sqrt{1 - x^2}} \, dx \approx \frac{\pi}{n} \sum_{i=1}^n f\left( \cos\left( \frac{(2i-1)\pi}{2n} \right) \right)
        $$
    \end{block}

    \begin{block}{Caractéristiques}
        \begin{itemize}
            \item Tous les poids sont égaux : $w_i = \pi/n$
            \item \textbf{Avantage :} Gère naturellement les singularités en $x = \pm 1$
            \item \textbf{Usage :} Problèmes avec comportement oscillant ou singulier
            \item Points de quadrature : zéros du polynôme de Chebyshev
        \end{itemize}
    \end{block}

    \textbf{Exemple test :} $\int_{-1}^1 \frac{x^4}{\sqrt{1-x^2}} \, dx = \frac{3\pi}{8}$
\end{frame}

\begin{frame}{Résultats - Gauss-Chebyshev {\small $\int_{-1}^{1} \frac{x^4}{\sqrt{1-x^2}} dx = \frac{3\pi}{8}$}}
    \begin{figure}
        \centering
        \includegraphics[width=0.48\textwidth]{figure/integral/gauss_chebyshev/gauss_chebyshev_erreur.png}
        \hfill
        \includegraphics[width=0.48\textwidth]{figure/integral/gauss_chebyshev/gauss_chebyshev_temps.png}
    \end{figure}

    \textbf{Interprétation :}
    \begin{itemize}
        \item Poids $\frac{1}{\sqrt{1-x^2}}$ géré naturellement : pas de singularité numérique
        \item Fonction polynomiale $\Rightarrow$ exactitude pour $n \geq 3$ (degré $2n-1 = 5$)
        \item Calcul très rapide : points explicites $x_i = \cos\left(\frac{(2i-1)\pi}{2n}\right)$
    \end{itemize}
\end{frame}

\begin{frame}{Règle de Simpson composite}
    \begin{block}{Formule}
        Pour $n$ sous-intervalles (n pair), avec $h = \frac{b-a}{n}$ :
        $$
        \int_a^b f(x) \, dx \approx \frac{h}{3} \left[ f(a) + 4\sum_{i \text{ impair}} f(x_i) + 2\sum_{i \text{ pair}} f(x_i) + f(b) \right]
        $$
    \end{block}

    \begin{block}{Caractéristiques}
        \begin{itemize}
            \item \textbf{Ordre :} $\mathcal{O}(h^4)$ pour $f \in C^4$
            \item \textbf{Principe :} Interpolation parabolique par morceaux
            \item \textbf{Avantage :} Facile à implémenter, précision acceptable
            \item \textbf{Limitation :} Convergence algébrique (plus lente que Gauss)
        \end{itemize}
    \end{block}
\end{frame}

\begin{frame}{Résultats - Simpson {\small $\int_0^{\pi} \sin(x) dx = 2$}}
    \begin{figure}
        \centering
        \includegraphics[width=0.48\textwidth]{figure/integral/simpson/gauss_legendre_erreur.png}
        \hfill
        \includegraphics[width=0.48\textwidth]{figure/integral/simpson/gauss_legendre_temps.png}
    \end{figure}

    \textbf{Interprétation :}
    \begin{itemize}
        \item Convergence algébrique $\mathcal{O}(h^4)$ confirmée graphiquement (pente $-4$ en log-log)
        \item Plus lent que Gauss-Legendre pour même précision
        \item Avantage : simplicité d'implémentation, points uniformément espacés
    \end{itemize}
\end{frame}

\begin{frame}{Intégration par spline quadratique}
    \begin{block}{Principe}
        \begin{enumerate}
            \item Interpolation de $f$ par splines quadratiques sur les points $x_i$
            \item Intégration analytique de chaque segment parabolique
            \item Sommation des contributions de tous les segments
        \end{enumerate}
    \end{block}

    \begin{block}{Caractéristiques}
        \begin{itemize}
            \item \textbf{Flexibilité :} S'adapte à la géométrie de la fonction
            \item \textbf{Régularité :} Continuité $C^1$ de la spline
            \item \textbf{Avantage :} Bonne performance sur fonctions irrégulières
            \item \textbf{Coût :} Légèrement supérieur (construction de la spline)
        \end{itemize}
    \end{block}
\end{frame}

\begin{frame}{Résultats - Spline quadratique {\small $\int_{-1}^{1} \frac{1}{1+25x^2} dx = \frac{2}{5}\arctan(5)$}}
    \begin{figure}
        \centering
        \includegraphics[width=0.48\textwidth]{figure/integral/spline/spline_erreur.png}
        \hfill
        \includegraphics[width=0.48\textwidth]{figure/integral/spline/spline_temps.png}
    \end{figure}

    \textbf{Interprétation :}
    \begin{itemize}
        \item Fonction de Runge : oscillations rapides difficiles pour méthodes polynomiales
        \item Spline s'adapte localement à la géométrie $\Rightarrow$ robustesse accrue
        \item Convergence $\mathcal{O}(h^3)$ avec coût calculatoire modéré
    \end{itemize}
\end{frame}

\begin{frame}{Comparaison des méthodes d'intégration}
    \begin{table}
        \centering
        \tiny
        \begin{tabular}{|l|c|c|c|}
        \hline
        \textbf{Méthode} & \textbf{Domaine} & \textbf{Convergence} & \textbf{Usage optimal} \\
        \hline
        Gauss-Laguerre & $[0, +\infty)$ & Exponentielle & Intégrales semi-infinies \\
        Gauss-Legendre & $[a, b]$ & Exponentielle & Fonctions régulières \\
        Gauss-Chebyshev & $[-1, 1]$ & Exponentielle & Singularités aux bords \\
        Simpson & $[a, b]$ & $\mathcal{O}(h^4)$ & Implémentation simple \\
        Spline quad. & $[a, b]$ & $\mathcal{O}(h^3)$ & Fonctions irrégulières \\
        \hline
        \end{tabular}
        \caption{Synthèse comparative des méthodes}
    \end{table}

    \vspace{0.2cm}

    \begin{alertblock}{Recommandation}
        Choisir la méthode selon :
        \begin{itemize}
            \item Le domaine d'intégration (fini, semi-infini)
            \item La régularité de la fonction
            \item La précision requise vs. coût calculatoire
        \end{itemize}
    \end{alertblock}
\end{frame}


% ========================================
% SECTION 5 : CONCLUSION
% ========================================
\section{Conclusion}

\begin{frame}{Conclusion générale}

\small

\begin{alertblock}{Synthèse du projet}
    Ce projet a permis une étude approfondie de \textbf{huit méthodes numériques fondamentales}, couvrant deux domaines centraux du calcul scientifique : la résolution numérique des équations différentielles ordinaires et le calcul approché d’intégrales définies.
\end{alertblock}

\vspace{0.2cm}

\begin{block}{Apports majeurs du travail}
    \begin{itemize}
        \item \textbf{Apport théorique :} Consolidation des notions clés d’ordre de convergence, de stabilité numérique et d’analyse des erreurs.
        \item \textbf{Apport expérimental :} Mise en évidence des comportements numériques par confrontation systématique aux solutions analytiques et aux résultats de référence.
        \item \textbf{Apport méthodologique :} Élaboration d’une démarche rigoureuse permettant de choisir une méthode adaptée en fonction des caractéristiques du problème, du niveau de précision recherché et du coût de calcul.
    \end{itemize}
\end{block}

\vspace{0.2cm}

\begin{center}
    \textit{« En analyse numérique, il n’existe pas de méthode universellement optimale.}\\
    \textit{Le choix d’une approche doit résulter d’un compromis raisonné}\\
    \textit{entre précision, stabilité numérique et ressources computationnelles. »}
\end{center}

\normalsize
\end{frame}

\begin{frame}
    \begin{center}
        \Huge \textbf{Merci pour votre attention}

        \vspace{1cm}

        \Large \textbf{Questions ?}
    \end{center}
\end{frame}

\end{document}